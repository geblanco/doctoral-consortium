\documentclass{beamer}

\usepackage{hyperref}
\usepackage{tikz}

%\setbeameroption{show only notes}
% \usetheme{material}

\usetheme[progressbar=frametitle, numbering=fraction]{metropolis}
% \usetheme[sectionpage=none,progressbar=frametitle, numbering=fraction]{metropolis}  

\title{%
  \textbf{Question Answering and \\Lifelong Learning} \\%
  {\small Programme: Intelligent Systems}
}
\author{%
  Guillermo Echegoyen\\%
  {\color{gray}\footnotesize Advised by: \\$\-$ Anselmo Pe\~nas \\$\-$ Alvaro Rodrigo}\\%
}

% \institute{UNED ETS. Inform\'atica}
\date{}

\titlegraphic{%
  \flushright\includegraphics[width=2cm,height=2cm]{logo.png}
}

\addtobeamertemplate{frametitle}{}{%
  \begin{tikzpicture}[remember picture,overlay]
    \node[anchor=north east,yshift=2pt] at (current page.north east) {\includegraphics[height=0.8cm]{logo}};
  \end{tikzpicture}
}

\setbeamercolor{frametitle}{fg=brown,bg=brown!20}
\setbeamertemplate{page number in head/foot}{}
\setbeamertemplate{frame footer}{Doctoral Symposium, June 2020}

\begin{document}
\maketitle

\begin{frame}{Table of contents}
  \setbeamertemplate{section in toc}[sections numbered]
  \tableofcontents[hideallsubsections]
\end{frame}

\section{Who am I}
\begin{frame}{Who am I}
  \begin{columns}[T,onlytextwidth]
    \column{0.45\textwidth}
    \begin{alertblock}{Background}
      \begin{itemize}
        \item Software Engineering
        \item AI Master
      \end{itemize}
    \end{alertblock}
    \column{0.45\textwidth}
    \begin{alertblock}{Where to find me}
      \begin{itemize}
        \item \href{https://geblanco.github.io}{Resum\'e}
        \item \href{https://scholar.google.es/citations?user=5XnvPusAAAAJ&hl=en}{Google Scholar}
      \end{itemize}
    \end{alertblock}
  \end{columns}
  \begin{alertblock}{Currently}
    \begin{itemize}
      \item International project LIHLITH-KIQA (PI: Anselmo Pe\"nas). Question Answering that evolve over time.
      \item First year PhD in Lifelong Learning Question Answering.
    \end{itemize}
  \end{alertblock}
\end{frame}

\section{Introduction}
% ToDo := What is active learning
\begin{frame}{Introduction: Lifelong Learning}
  % \metroset{block=fill}
  \begin{block}{Lifelong Learning}
    \vspace{0.2cm}
    \textit{Framework to study how to enable systems to evolve over time, integrating new knowledge with previous one and adapting to new tasks.} \par

    Can be applied to multiple learning paradigms and techniques:
    \begin{itemize}
      \item Active Learning
      \item Reinforcement Learning
      \item Transfer Learning
      \item Knowledge acquisition
    \end{itemize}
  \end{block}
\end{frame}

\begin{frame}{Introduction: Question Answering}
  % \metroset{block=fill}
  \begin{block}{Question Answering}
    \vspace{0.2cm}
    \textit{Systems that automatically answer questions posed by humans in a natural language} \\
    \begin{alertblock}{Focus}
      \begin{itemize}
        \item Based on Knowledge Graphs
        \item Based on other resources: Set of documents, search engines
      \end{itemize}
    \end{alertblock}
    Applications:
    \begin{itemize}
      \item Question Answering Systems are everywhere, from personal assistants to chatbots and IT ticket management.
    \end{itemize}
  \end{block}
\end{frame}

\note[itemize]{%
  \item Specifically, we work with 
  \item KG QA
  \item Extractive QA
  \item Reading comprehension QA
}

\begin{frame}{Thesis Frame}
  Most QA systems:
  \begin{itemize}
    \item Work under a closed world assumption, ranking plausible answers
    \item Learn only during training, becoming stale over time
    \item Fail to sustain performance in production environments
  \end{itemize}
  We focus on Lifelong Learning Question Answering systems, that:
  \begin{itemize}
    \item Evolve over time
    \item Integrate new knowledge with previous one
    \item Adapt to new tasks
  \end{itemize}
\end{frame}

\section{Research Plan}
% ToDo := Review RQ 3 (effectiveness)
\begin{frame}{Research Plan: Objectives}
  \begin{alertblock}{Objectives}
    \begin{itemize}
      \item Develop QA systems sustainable in production environments
      \item Develop QA systems capable of detecting knowledge gaps
      \item Develop strategies to fill these gaps
      \item Enable QA systems to evolve over time
      \item Enable QA systems to adapt to new tasks.
    \end{itemize}
  \end{alertblock}
\end{frame}

\note[itemize]{%
  \item Information evolve over time (dates, dead) there is a need to evolve over time.
}

\begin{frame}{Research Plan: Open Questions}
  \begin{alertblock}{Research Questions}
    \begin{itemize}
      \item How to detect unanswerable questions given the available knowledge
      \item How to incorporate new knowledge with previous one
      \item How to asses QA systems effectiveness
      \item How to adapt QA systems to new tasks
    \end{itemize}
  \end{alertblock}
\end{frame}

\note[itemize]{%
  \item There are plenty of research questions, but we will restrain to those.
}

% internship?
\begin{frame}{Research Plan: First Year}
  \alert{Goal}: Study the state-of-the-art
  \vspace{0.2cm}
  \begin{itemize}
    \item Related to Knowledge Graphs: \\
      {\small Guillermo Echegoyen, Álvaro Rodrigo, Anselmo Peñas (2019). \textbf{Benchmarking Entity Linking for Question Answering over Knowledge Graphs}. Sociedad Espa\~nola de Procesamiento de Lenguaje Natural (PLN). Volume 63, pages 121-128. ISSN: 19897553, 11355948. DOI: 10.26342/2019-63-13. \cite{Echegoyen2019}}
  \end{itemize}
  \begin{block}{Outcomes}
    \begin{itemize}
      \item 6 Entity Linking evaluation datasets.
      \item EL has a higher impact over QA systems than usually though.
    \end{itemize}
  \end{block}
\end{frame}

% ToDo := Add Outcomes?
\begin{frame}{Research Plan: First Year}
  \alert{Goal}: Study the state-of-the-art
  \vspace{0.2cm}
  \begin{itemize}
    \item Related to NLP and Dialogue Systems: \\
      {\small Jan Deriu, Alvaro Rodrigo, Arantxa Otegi, Guillermo Echegoyen, Sophie Rosset, Eneko Agirre, Mark Cieliebak (2020). \textbf{Survey on evaluation methods for dialogue systems}. Artificial Intelligence Review. DOI: 10.1007/s10462-020-09866-x. \cite{jan_survey_2020}}
  \end{itemize}
  \begin{block}{Outcomes}
    \begin{itemize}
      \item Trend towards end-to-end, more cryptic systems, based on large amounts of data.
      \item Trend towards less human involvement regarding evaluation. 
    \end{itemize}
  \end{block}
\end{frame}

\begin{frame}{Research Plan: First Year}
  \alert{Goal}: Study the state-of-the-art
  \vspace{0.2cm}
  \begin{itemize}
    \item Related to Lifelong Learning (WIP): \\
      {\small Guillermo Echegoyen, Álvaro Rodrigo, Anselmo Peñas (2019). \textbf{Study of a Lifelong Learning Scenario for Question Answering}}
  \end{itemize}
  \begin{block}{Outcomes}
    \begin{itemize}
      \item Transfer learning between different extractive QA datasets.
      \item Catastrophic forgetting is a big concern.
    \end{itemize}
  \end{block}
\end{frame}

\begin{frame}{Research Plan: First Year}
  \vspace{-0.1cm}
  \alert{Goal}: Study the state-of-the-art
  \vspace{0.1cm}
  \begin{itemize}
    \item Position papers (shared task proposal): \\
      {\small Anselmo Peñas, Mathilde Veron, Camille Pradel, Arantxa Otegi, Guillermo Echegoyen and Alvaro Rodrigo (2019). \textbf{Continuous Learning for Question Answering. Dialogue Systems and Lifelong Learning}. Increasing Naturalness and Flexibility in Spoken Dialogue Interaction: 10th International Workshop on Spoken Dialogue Systems, Lecture Notes in Electrical Engineering, Springer. ISSN: 1876-1100 \cite{penas2019continuous}}
  \end{itemize}
  \begin{block}{Outcomes}
    \begin{itemize}
      \item Task: Determine mappings from utterances to a KG and means to enrich the KG.
    \end{itemize}
  \end{block}
\end{frame}

\begin{frame}{Research Plan: First Year}
  \vspace{-0.2cm}
  \alert{Goal}: Study the state-of-the-art
  \vspace{0.1cm}
  \begin{itemize}
    \item Position papers (LL Evaluation collection): \\
      {\small Mathilde Veron, Anselmo Peñas, Guillermo Echegoyen, Somnath Banerjee, Sahar Ghannay, and Sophie Rosset (2020). \textbf{A Cooking Knowledge Graph and Benchmark for Question Answering Evaluation in Lifelong Learning scenarios}. Natural Language in the Database and Information Systems (NLDB), Lecture Notes in Computer Science, Springer 2020. Volume 12089, pags. 94-101. ISSN: 0302-9743 \cite{veron_nldb_2020}}
  \end{itemize}
  \begin{block}{Outputs}
    \begin{itemize}
      \item Cooking Domain Knowledge graph
      \item Cooking Domain Question Answering dataset
    \end{itemize}
  \end{block}
\end{frame}

\begin{frame}{Research Plan: First Year}
  \vspace{-0.2cm}
  \alert{Goal}: Study the state-of-the-art
  \vspace{0.1cm}
  \begin{itemize}
    \item Lifelong Learning and transfer learning: \\
      {\small Guillermo Echegoyen, Álvaro Rodrigo, Anselmo Peñas (en prensa). \textbf{Cross-lingual Training for Multiple-Choice Question Answering}. Sociedad Española de Procesamiento de Lenguaje Natural (PLN). Accepted in March 2020, Volume 64, September. ISSN: 1989-7553. \cite{echegoyen_sepln_2020}}
  \end{itemize}
  \begin{block}{Outputs}
    \begin{itemize}
      \item Exams dataset: Difficulty affects human and machines in a similar way.
      \item QA systems can be transferred to different tasks with acceptable performance, even in different languages.
    \end{itemize}
  \end{block}
\end{frame}

\begin{frame}{Research Plan: Second Year}
\end{frame}

\begin{frame}{Research Plan: Third \& Fourth Years}
\end{frame}

\section{Conclusions}
\begin{frame}{First Year Conclusions}
  \begin{alertblock}{Experimentation is hard:}
    \begin{itemize}
      \item Money for value
      \item Interpretability is difficult
      \item Catastrophic forgetting is hard to overcome
      \item We are far from real world QA systems
      \item Ideal laboratory settings are far from real world use cases
    \end{itemize}
  \end{alertblock}
\end{frame}

\note[itemize]{%
  \item Deep Learning aims for a number (accuracy). 
  \item Deployed systems face many more problems and inconsistencies than laboratory environments.
}

\section{Related Challenges}
\begin{frame}{Related Challenges}
  \alert{\Large Trend towards Deep Learning}
  \begin{itemize}
    \item Large code base
    \item Programming GPUs
    \item Experiments are costly
    \item Cloud computing
    \item Results interpretation not always clear
  \end{itemize}
\end{frame}

\note[itemize]{%
  \item Many libraries
  \item Lots of algorithms
  \item AWS, GCP
}

% outro: ToDo := Add cool quote
\begin{frame}
  \begin{center}
    \Huge Thank you! \\
    \huge Questions? \\
    \Large Guillermo Echegoyen
  \end{center}
\end{frame}

% ToDo := Sort references by appearance
\begin{frame}[allowframebreaks]{References}
  \bibliography{library}
  \bibliographystyle{abbrv}
\end{frame}

\end{document}
